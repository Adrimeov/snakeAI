%%%% ijcai19-multiauthor.tex

\typeout{IJCAI-19 Multiple authors example}

% These are the instructions for authors for IJCAI-19.

\documentclass{article}
\pdfpagewidth=8.5in
\pdfpageheight=11in
% The file ijcai19.sty is NOT the same than previous years'
\usepackage{ijcai19}

% Use the postscript times font!
\usepackage{times}
\usepackage{soul}
\usepackage{url}
\usepackage[hidelinks]{hyperref}
\usepackage[utf8]{inputenc}
\usepackage[small]{caption}
\usepackage{graphicx}
\usepackage{amsmath}
\usepackage{booktabs}
\urlstyle{same}
\usepackage{biblatex}
\addbibresource{ijcai19.bib}

% the following package is optional:
%\usepackage{latexsym} 

% Following comment is from ijcai97-submit.tex:
% The preparation of these files was supported by Schlumberger Palo Alto
% Research, AT\&T Bell Laboratories, and Morgan Kaufmann Publishers.
% Shirley Jowell, of Morgan Kaufmann Publishers, and Peter F.
% Patel-Schneider, of AT\&T Bell Laboratories collaborated on their
% preparation.

% These instructions can be modified and used in other conferences as long
% as credit to the authors and supporting agencies is retained, this notice
% is not changed, and further modification or reuse is not restricted.
% Neither Shirley Jowell nor Peter F. Patel-Schneider can be listed as
% contacts for providing assistance without their prior permission.

% To use for other conferences, change references to files and the
% conference appropriate and use other authors, contacts, publishers, and
% organizations.
% Also change the deadline and address for returning papers and the length and
% page charge instructions.
% Put where the files are available in the appropriate places.

\title{Entrainement d'un agent virtuelle par Q-learning pour le jeu Snake }

\author{
Samuel Ferron\and
Jean-Frédéric Fontaine\and
Noboru Yoshida
\affiliations
Department de génie logiciel, Polytechnique Montréal, Canada\\
\emails
\{samuel.ferron, jean-frederic.fontaine, noboru.yoshida\}@polymtl.ca
}

\begin{document}

\maketitle

\begin{abstract}
This short example shows a contrived example on how to format the authors' information for {\it IJCAI--19 Proceedings} using \LaTeX{}.
\end{abstract}

\section{Introduction}

Noboru

\section{Revue de la littérrature}
 
Noboru

\section{Explication théorique }
Généralement, les algorithmes d’apprentissages machines dépendent d’un grande quantité de donnée qui ont été étiquetées à la main afin de pouvoir faire des prédictions de qualité. Cependant, dans le contexte de l'apprentissage par renforcement, l’idée est plutôt de permettre à un l’algorithme, c’est-à-dire un agent, d’apprendre par lui-même les règles qui définissent le succès ou l’échec d’une tâche quelconque. Ainsi, on fournit à l’agent une représentation de l’état dans lequel il se trouve et une récompense dépendamment de la qualité des actions que l’agent entreprend. Qui plus est, nous avons implémenté l’algorithme présenté dans le papier de Mnih et al. \cite{DBLP:journals/corr/MnihKSGAWR13} à la seule différence que nous n'avons pas donné comme entrée au réseaux la représentation  visuelle mais plûtot une représentation simplifiée sous forme de vecteur one-hot de certain paramètres de la partie. Cette approche nous as permis de simplifier la période d'apprentissage compte tenu du matériel à notre disposition. Cette approche allégée est inspiré du blog par Mauro Comi \cite{comi_2020}. 


\subsection{Algorithme d'apprentissage}



Ainsi, 
\section{Explication de notre implémentation}

Sam

\section{Résultats}

Sam

\section{Améliorations possible}

Sam

\section{Conclusion}

JF

\printbibliography
\end{document}

